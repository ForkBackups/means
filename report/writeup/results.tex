\section{Results and Discussion} \label{results}


\subsection{Performance}\label{performance}
The temporal dynamic of a molecular system can be described by the \gls{cme}.
%~ \cite{}
\gls{gssa}\cite{gillespie_general_1976} produces individual realisations of the \gls{cme}.
In order to obtain accurate estimates of the average dynamic within a population of cell, it it however necessary to perform multiple (often more than $10^4$) simulations.
Despite recent effort \cite{niemi_efficient_2011,dittamo_optimized_2009,komarov_accelerating_2012} to provide fast implementation of this algorithm, computation remains extremely expensive.
This is critical when performing, for instance, parameter inference.
This limitation led to the development of approximations such as \gls{lna}\cite{komorowski_bayesian_2009} and \gls{mea}\cite{ale_general_2013} which can perform in a more reasonable time.

Since the only advantage of \gls{mea} over \gls{gssa} is computational speed, it was paramount to provide an efficient an implementation of \gls{mea}.
In this section, we show that symbolic computations can be limiting for \gls{mea}.
Then, we explain how we have optimised them, and finally increase performance by several order of magnitudes over to the original \mat{} implementation.
In addition, ...\\
TODO @saul NUMERICAL EVAL/solvers.


\subsubsection{Symbolic expressions}


\gls{mea} involves derivation of a system \gls{ode}s from a model.
This procedure\cite{ale_general_2013}, involves lengthy symbolic calculations.
Even for very simple models (\eg{} three species, five reactions), they cannot be realised manually.
The number $n$ of modelled moments (\ie{} the number of generated \gls{ode}s), for a system with $s$ species and up to moments of order $o$ is defined by:\\
\begin{equation}
n={{s+o-2} \choose {s}} -1
\end{equation}
As a consequence, the complexity of the calculation is predicted to increase exponentially with the number of species in the system and the maximal order of moments.

In order to perform symbolic computations, we have used \sympy{} \cite{sympy_development_team_sympy:_2014}; an efficient \py{} library.
After implementation of the different features (TODO see section closure/...)
%~ see \ref{}...
, we have endavoured to optimise the processing speed.
In a first place, we identified significant bottlenecks using \py{} profiling tools.
Then, we designed specific ways to improve speed incrementally without changing the resulting \gls{ode}s.
%see unitests
Figure%~\ref{}
~shows the cumulative effects of different optimisations.


The first step involved removing the expression simplification heuristic.
In the original code\footnote{both from the publication and last year's MSc project}, the right-hand-side equations were simplified in order to produce shorter text file results.
However, this was slow and did not benefit subsequent simulations and inference.
For large expressions, simplification had also had an large memory footprint and was likely to fail.
This optimisation resulted in a 
%...
TODO reduction of ....

The next bottleneck was the choice of substitution functions.
As a part of \gls{mea}, it is necessary to replace raw moment symbols by expressions depending on central moments.
Performing substitution can be done using the \texttt{substitute()} function from \sympy, but this is designed to substitute expressions by other expressions.
In most cases, we only had to substitute atomic symbols by expression.
For this purpose, the  \texttt{xreplace()} function was a much more appropriate alternative which resulted in ........TODO improvement.

In the original implementation, a matrix of central moment expression of size $(n-s) \times (n-s + 1)$ is directly generated when the default closure is applied.
However, when using a parametric closure, a matrix of size $(n_2-s) \times (n_2-s + 1)$, where $n_2={{s+o-2 \mathbf{+1}} \choose {s}} -1$, was generated.
The $n_2 - n$ rows corresponding to higher-order moments have then to be deleted.
In contrast, out implementation generates a $(n-s) \times (n_2-s + 1)$ matrix regardless of the closure method.
In addition to improve code readability, consistency and flexibility \footnote{see QG's individual report}, this improved performance for cases where closure is applied while keeping the default closure computation fast.

Another simple way to improve computation time was to remove calls to the function \texttt{solve()} which was only used in straightforward cases (\eg{} solving: $a + 2b = c$ for $a$).
It was therefore much more efficient to use simple arithmetic to find solution.

Finally, partial derivation of expression over several variables and order is extensively performed during the approximation.
Generally, these type of differentiations can be simplified several differentiation of first order:
\begin{equation}
\frac{\partial{} ^ 2 f(x,y)}{\partial x \partial y}  =
\frac{\partial{} \frac{\partial{} f(x,y)}{\partial x}}{\partial y} =
\frac{\partial{} \frac{\partial{} f(x,y)}{\partial{} y}}{\partial{} x}
\end{equation}
One advantage, is that, when needing to calculate two derivatives such as:  $\frac{\partial{} ^ 2 f(x,y)}{\partial{} x \partial{} y}$ and $\frac{\partial{} ^ 2 f(x,y)}{\partial{} x^2}$, 
one can precompute $\frac{\partial{} f(x,y)}{\partial{} x}$ and use it for both calculation.
In our implementation, we have use a procedure known as \emph{memoization} which, briefly, permits to store the results of a function call in an associative array.
Then, the next time this function is called with the same arguments, it will return the stored results instead of recomputing it.
TODO this resulted in !!

