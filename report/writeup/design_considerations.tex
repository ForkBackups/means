\section{Approach to Software Design}
\label{sec:design_considerations}

Our software development has been heavily inspired by the Agile methodology\cite{_manifesto_????}. 
The key principles of this methodology have been neatly summarised in a blog post by Kelly Waters\cite{_what_????}:

\blockquote{
    [The] characteristics that are common to all agile methods, and the things that I think make agile fundamentally different to a more traditional waterfall approach to software development [..] are:
    
    \begin{enumerate}
        \item Active user involvement is imperative 
        \item The team must be empowered to make decisions 
        \item Requirements evolve but the timescale is fixed 
        \item Capture requirements at a high level; lightweight \& visual 
        \item Develop small, incremental releases and iterate 
        \item Focus on frequent delivery of products 
        \item Complete each feature before moving on to the next 
        \item Apply the 80/20 rule 
        \item Testing is integrated throughout the project lifecycle – test early and often 
        \item A collaborative \& cooperative approach between all stakeholders is essential 
    \end{enumerate}
}

We have particularly focused on capturing requirements well and tracking when they change. 
Delivering new content in small iterations, one feature at a time, getting feedback from our stakeholders and potential users as fast as possible.

\subsection{Requirements Gathering}

We spent the initial meetings with the supervisors capturing the main requirements for this project. These requirements were captured in the form of user stories, that describe an imaginary scenario in which the software is used. 
The relatively free-form of story telling allows the requirement gathering process to focus on the user's point of view, rather than getting slowed down with the implementation details. 
After all, the end user does not need to care about the implementation, it is our job, as software engineers, to make sure it is appropriate for the requirements specified.
This is very much what the \emph{Capture requirements at a high level} statement in the previous list of agile key concepts stands for.

We were able to write down four user stories, listed below:

\blockquote{As a researcher studying biological systems, I would like to perform a variety of simulations, with different parameter sets, and compare the results of these simulations.}

\blockquote{As a researcher studying the approximation method, I would like to be able to compare the changes to the approximation quality, when the approximation parameters (i.e. the closure method) are changed.}

\blockquote{As a researcher studying biological systems, I would like to be able to infer the set of parameters of the model given the data, but this idea is still not very popular in the field and I am not sure if its feasible.}

\blockquote{As a systems biologist, I do not care about the input format of the data, as I am computer-savvy enough to write a script to match the input format.}

The vagueness of these user stories show that there is no clearly established pipeline for the use of this software. 
For example, when we asked to elaborate on  what was meant by ``compare the results", we were not able to get any clear description of this, as it depends very much on the specific problem being investigated.
And this is very much expected, given the overall exploratory nature of the method. 

What this use case tells us, however, is that we need to be very careful to not impose any constraints that would limit the creativity of the users and support as many use cases as possible.

We believe that this freedom for the user is not possible to achieve from a command line application as it, almost by definition, imposes a specific pipeline through the process. Similarly, we believe that immediate feedback, rather than the feedback at the end of the computational pipeline, is necessary for data analysis to occur efficiently. 
We think that interactivity is the key to solving the problems described above.
That is why we chose an interaction-centered approach in the design of this package, and made sure it integrates nicely with 
 \todo{how to quote this again?}IPython notebook environment.
