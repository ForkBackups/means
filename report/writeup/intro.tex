\section{Introduction} \label{intro}
\subsection{Moment Closure Methods}
To model the changes in concentrations using \gls{mea}, higher order moment are required to express the lower order moments, i.e. the $ith$ moment depends on the $(i+1)th$ moment. This makes the expressions for moment infinitely long, and the \glspl{ode} impossible to solve. Therefore, a closure method is necessary to approximate the expressions for higher order moments either by assuming the higher order moments are zeros, or by assuming a parametric distribution of the population. Moment closure is particularly useful for stochastic population models, as it leads to a set of \emph{"closed"} solvable differential equations to model the system. \todo{cite Moment-closure approximations for mass-action models}

Assuming zeros of all higher order moments reduces the computation effort, as the higher order moments are simply neglected. However, this method does exploit any parametric distribution for the population, resulting in inaccurate simulation if the type of distribution was known. 

In contrast, several parametric moment closure methods are available, where they estimate higher order moments using an assumed distribution type. This allows researchers to use the known distribution type to model the system more precisely. 

In \means, we have deployed both non-parametric and three parametric closure methods - Normal, Log-normal and Gamma closure. For each method, the data can be either univariate or multivariate. Univariate is used when only one species is modelled in a system, or when multiple species can be regarded as one species, for instance, in dimerisation reactions. 

Each of the parametric method is based on a therem. Normal closure uses Isserlis\rq , theorem or Wick\rq s theorem to compute higher order moments in terms of covariance matrix.\todo{cite On a formula for the product-moment coefficient of any order of a normal frequency distribution in any number of variables} The theorem assumes that the population distribution for all the species in the system is a zero mean multivariate normal random vector. It approximates the even order moments as the sum of products of all the possible pairs of partitions, whereas the odd order moments are defined to be zeros. For example, the fourth order moments $\mathrm{E}[x_1x_2x_3x_4] = \mathrm{E}[x_1x_2]\ \mathrm{E}[x_3x_4] + \mathrm{E}[x_1x_3]\ \mathrm{E}[x_2x_4]+\ \mathrm{E}[x_1x_4]\ \mathrm{E}[x_2x_3]$. In the univariate case, the therem works by simply setting the covariances and to zeros. 

Log-normal closure was first documented by Crow. \todo{cite the book Theory and Applications} Similarly, Log-normal closure also estimates higher order moments in terms of log-covariance matrix, but the odd order moments are estimated based on the algorithm described in  Singh \& Hespanha, 2006. \todo{cite Lognormal Moment Closures for Biochemical Reactions} The advantage of using Log-normal closure, compared with Normal closure, is that its skewness and non-negative properties better describe populations in reality.\todo{cite Novel moment closure approximations in stochastic epidemics}

Current definitions of Gamma distribution result in different types of Gamma moment closure methods.\todo{cite On a multivariate-gamma and On a multivariate gamma distribution} Both definitions of multivariate Gamma distribution use the parameters  to express the first and second order moments, and any higher order moments can be derived from multinomial formula. \todo {cite Eszter unpublished} One of the limitations of Gamma closure is incorrect modelling for systems with negatively correlated species. Moreover, prior knowledge about the population distribution is critical in choosing the parameter values for Gamma distribution. 

Using \means, we intend to further explore the quality of these closure methods using several bio-models, and the results are analysed in Results and Discussion.
 
blabe bla \cite{ale_general_2013}
